% template.tex - template for matrix2tikz output file
%
% Other files required: none
% Subfunctions: none
% MAT-files required: none
%
% See also: 
% Author: Hannes Dillinger
% Institute for Biomedical Engineering, ETH Zurich
% email: dillinger@biomed.ee.ethz.ch
% Website: http://www.
% April 2019; Last revision: 17-April-2019
%
% DO NOT EDIT THIS FILE
%
%------------- BEGIN CODE ---------------

\pgfplotsset{
	matrix rows/.store in=\matrixrows,
	matrix rows=#1,
	matrix cols/.store in=\matrixcols,
	matrix cols=#2,
	parxmin/.store in=\parxmin,
	parxmin=#3,
	parxmax/.store in=\parxmax,
	parxmax=#4,
	parymin/.store in=\parymin,
	parymin=#5,
	parymax/.store in=\parymax,
	parymax=#6,
}

\begin{tikzpicture}
	\pgfplotstableread{plots/matlab/#7.txt}\datatable; 
	\begin{axis}[
	%    
	% define these lengths in your preamble using
	% \global\setlength\figbigwidth{0.98\linewidth}
	% \global\setlength\figbigheight{0.25\textheight}
	%    	
	width=\figbigwidth,
	height=\figbigwidth,
	%        
	% colormaps
	colormap/hot,
	%		colormap/viridis,
	%
	% colorbar
	colorbar sampled,
	%       colorbar horizontal,
	samples=255,
	colorbar style={
		samples=255,
		%        	height=10cm,
		%        	ytick={295.1111,295.1121},
		yticklabel style={/pgf/number format/.cd,fixed,precision=9},
	},
	%    
	axis on top,
	xmin=\parxmin,
	xmax=\parxmax,
	ymin=\parymin,
	ymax=\parymax,
	enlarge x limits={rel=0.5/\matrixcols},
	enlarge y limits={rel=0.5/\matrixrows},
	point meta min=0,
	point meta max=0.0022,
	]

	